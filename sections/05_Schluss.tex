\chapter{Zusammenfassung und Ausblick}{\label{ch:final}}
Die vorliegende Masterarbeit stellte eine umfassende Untersuchung und Analyse von Phasenübergängen unter Verwendung der \(\Gamma\)-Konvergenz und Regularitätstheorie dar. Ziel dieser Arbeit war es, Schritt für Schritt einen tiefgreifenden Einblick in das \\Forschungsgebiet zu geben und kleinere ergänzende Erkenntnisse zu gewinnen. Durch die systematische und sehr exakte Herangehensweise konnten wichtige Fragestellungen beleuchtet werden, ohne, dass dabei zusammenhängendes Verständnis verloren ging.

Die Arbeit begann mit einer eingehenden Literaturrecherche, um den aktuellen Stand der Forschung zu erfassen und eine solide Basis für die eigene Untersuchung zu schaffen. Diese Recherche umfasste hauptsächlich grundlegende Eigenschaften der \(\mathcal{BV}\)-Theorie und \(\Gamma\)-Konvergenz.

Im Verlauf der Arbeit wurden verschiedene variationelle Modelle betrachtet, die für sich stehend jeweils eine eigene Beweisstrategie erforderten. Insbesondere der \(\Gamma\)-Limsup Fall ist bei der Untersuchung von \(\Gamma\)-Konvergenz variationeller Modelle stets eine Herausforderung.

Ein bedeutender Teil dieser Masterarbeit war die Präzesierung vergangener \\Forschungsergebnisse. Oftmals ist das große Problem, das aus den Grundlagenvorlesungen im Mathematikstudium resuliert, dass man aus dem Rahmen der endlichdimensionalen reellen Analysis zu sehr an die herausragenden Eigenschaften von \(\mathbb{R}^n\) gewöhnt ist. Problematisch ist das deshalb, da oft nicht 100 prozentig heraussticht, warum bestimmte Prozesse, wie beispielsweise die Zerlegung der Eins, überhaupt möglich sind. Gerade unsere Natur ist sehr oft unschöner Struktur. Ein Beispiel zeigte diese Arbeit auf: Diskontinuierliche Probleme, wie Phasenübergänge, benötigen aus der gewöhnlichen reellen Analysis heraus zunächst merkwürdig wirkende Methoden. Wir hoffen, dass die Herangehensweise dieser Arbeit dazu beigetragen hat, die Vorstellung von allgemeiner Analysis zu erweitern.

An dieser Stelle möchte ich betonen, dass diese Masterarbeit nicht nur meine fachlichen Kenntnisse erweitert hat, sondern auch meine Fähigkeit zur eigenständigen wissenschaftlichen Arbeit gestärkt hat.

Die Arbeit soll Forschende im zugehörigen Bereich dazu animieren, die Ergebnisse fortzusetzen und vielleicht auch das große offene Problem aus \ref{subsec:soninva} zu lösen, dessen Problematik wir in \ref{sec:gendim} versucht haben, zu präzesieren.

Abschließend möchte ich mich bei meinem Betreuer Herrn Prof. Friedrich für wichtigen mathematischen Input, sowie meiner Familie, Freundin und Freunden für ihre Unterstützung und Ermutigung während meiner Studienzeit bedanken. Ihre Ratschläge und Anregungen waren stets eine Inspiration für mich.

Mit dem Abschluss dieser Masterarbeit schließe ich ein weiteres Kapitel meines akademischen Werdegangs ab. Ich freue mich darauf, das erworbene Wissen und die Erfahrungen in meiner beruflichen Laufbahn einzusetzen und weiterhin zur Entwicklung der Mathematik (in jeglicher Form) beizutragen.