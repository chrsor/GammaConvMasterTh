\chapter{Einleitung}
\begin{chapquote}{Ennio De Giorgi}
"If you can't prove your theorem, keep shifting parts of the conclusion to the assumptions, until you can."
\end{chapquote}
Die Variationsrechnung ist ein mathematisches Gebiet, das sich mit der Untersuchung von Minimierern/Maximierern von Funktionalen befasst. Sie spielt eine fundamentale Rolle in vielen Bereichen der Mathematik, Physik und Ingenieurswissenschaften. Der Begriff "Variation" bezieht sich hierbei auf die kleinen Änderungen oder Abweichungen eines Funktionals. Die Variationsrechnung untersucht, wie sich die Funktionswerte ändern, wenn man solche Variationen zulässt. Das Ziel besteht darin, die Funktion zu finden, für die die Änderung der Funktionswerte minimal oder maximal wird. Dies führt zur Formulierung von mathematischen Prinzipien, die als Variationsprinzipien bezeichnet werden.\\
Variationsrechnung hat Anwendungen in einer Vielzahl von Bereichen. In der Physik wird sie verwendet, um die Bewegungen von Teilchen und Systemen zu beschreiben, indem man die Aktion oder das Wirkungsfunktional minimiert. In der Ingenieurwissenschaft wird die Variationsrechnung zur Optimierung von Strukturen und Materialien eingesetzt. In der Ökonomie und den Sozialwissenschaften werden Variationen zur Maximierung oder Minimierung von Kosten, Nutzen oder anderen Zielfunktionen verwendet. Die Entwicklung der Variationsrechnung geht dabei auf das 17. Jahrhundert zurück, als Mathematiker wie Pierre de Fermat und Gottfried Wilhelm Leibniz begannen, Methoden zur Behandlung von Extremwerten zu entwickeln \cite{Goldstine1980}. Im Laufe der Zeit haben sich die Methoden der Variationsrechnung weiterentwickelt und sind zu einem wichtigen Werkzeug in der Analysis geworden.\\

In dieser Arbeit werden wir uns ein vielversprechendes, modernes Werkzeug der Variationsrechnung ansehen, die \(\Gamma\)-Konvergenz. Die \(\Gamma\)-Konvergenz wurde in den 1950er Jahren von Ennio De Giorgi und Giuseppe Buttazzo eingeführt und hat seitdem einen bedeutenden Einfluss auf verschiedene mathematische Disziplinen gehabt \cite{BioDeGiorgi}. Sie ermöglicht es, das Verhalten von Funktionalfolgen zu untersuchen, wenn sich die zugrunde liegende Energie- oder Kostenfunktion ändert. Die Idee hinter der \(\Gamma\)-Konvergenz besteht darin, die Konvergenz von Funktionenfolgen unter schwächeren Konvergenzeigenschaften zu betrachten als die übliche punktweise oder gleichmäßige Konvergenz. Anstatt Konvergenz der Funktionswerte zu betrachten, richtet sich das Augenmerk auf die Konvergenz der Energiefunktionale, die die Funktionen minimieren oder maximieren.